\documentclass[letterpaper, 12pt]{article}
\usepackage[pass]{geometry}
\usepackage{amsmath,amssymb}

\usepackage{amsmath}
\usepackage{graphicx}
\usepackage{caption}
\usepackage{graphicx}
\usepackage{float}
\usepackage{caption}
\usepackage{subcaption}
\restylefloat{table}
\usepackage{subcaption}



\makeatletter
\renewcommand{\fnum@figure}{Exhibit \thefigure}
\renewcommand{\fnum@table}{Exhibit \thefigure}
\captionsetup[subfigure]{labelformat=simple}
\renewcommand\thesubfigure{Panel \Alph{subfigure}:}

\makeatother

\begin{document}
 
\title{Blink Margin Engine}
\author{Proskurin Oleksandr}
\maketitle

\begin{abstract}
The paper describes a novel approach in calculating margin requirements for under-collateralized cross-chain leverage, considering DeFi-specific primitives such as liquidity pools, bridging and concentrated liquidity.
Key findings: 
\begin{enumerate}
  \item Blink Margin Engine is a centralized system designed to calculate portfolio risk using complex scenario analysis of the underlying asset prices.
  \item The Engine is designed to provide portfolio risk analytics in a multi-chain environment. 
  \item Blink Margin Engine is designed to become a universal middleware and standard in DeFi margin calculation for trading operations.
  \end{enumerate}
\end{abstract}


\section{Introduction}
The recent massive explosion of DeFi activity and innovation led to decentralized lending becoming a key part of an emerging financial system. Early DeFi success has been driven by over-collateralized lending protocols such as Aave and Compound, which enable borrowers to deposit collateral in one token and withdraw part of its value in another. However, as the system becomes increasingly mature, on-chain market participants, especially asset managers, lack access to undercollateralized loans, which provide more efficient capital allocation.

AAVE and Compound don't impose restrictions on operations done with leverage a user receives through their platform, which leads to the inefficient relative size of a loan in relation to the value of the collateral - \textbf {overcollateralized loan}. Asset managers are willing to accept constraints on operations done with leverage (trading-specific only) in exchange for an undercollateralized, capital-efficient loan which gives the ability to provide a better return on invested capital. On the other hand, undercollateralized loans, if liquidated incorrectly, impose creditors on serious risks, which is why a sophisticated system (\textbf {Margin Engine}) is needed to define when a user's loan should be liquidated. 

Secondly, the risks of several positions in a trader's portfolio may offset each other. For example, a portfolio consisting of ETH/USDC liquidity pool position hedged by short perpetual in ETH/USDT has much lower risk than an unhedged LP position in BTC/USDT or ETH/USDT. Margin calculations which account for total portfolio P\&L (\textbf {Cross Margin}) instead of segregated risk calculation for each position (\textbf {Isolated Margin}) improve capital and leverage efficiency for professional asset managers as maintenance margin requirements in Cross Margin is much lower. 

In this paper, we propose a unified framework for calculating cross-chain portfolio margin for under-collateralized loans taking into account De-Fi specifics:
  
  \begin{enumerate}
  \item Asset classes such as Liquidity Pools, Concentrated Liquidity LPs, and Liquid Staking Derivatives (LSD).
  \item Automated Market Makers (AMM) and Price Oracles.
  \item Gas costs and risks associated with Bridging.
  \end{enumerate}

 	


\section{Standard Portfolio Analysis of Risk (SPAN) and Standard Initial Margin Model (SIMM)}

The problem of multi-asset portfolio risk calculation is a well-researched and known issue solved in the traditional finance space. SPAN was developed in 1988 by Chicago Mercantile Exchange Inc. (CME) to effectively access risk on an overall portfolio basis. It calculates the potential loss in positions and sets this value as the initial margin payable by the firm holding the portfolio. As a result, SPAN provides for offsets between correlated positions and enhances margining efficiency. SPAN model has proven to be an efficient and robust way to calculate portfolio margin for multi-asset derivatives portfolios. It became the official Performance Bond mechanism of 54 exchanges and clearing organizations worldwide, making it the global standard for portfolio margining. 

SPAN model relies on scenario analysis and asset price simulations, taking into account their correlation structure. Given that, positions P\&Ls are calculated and aggregated on a portfolio level. Finally, SPAN groups financial instruments with the same underlying for analysis and calculates the portfolio's Value at Risk (VaR) metric, which serves as input for final margin calculations. 

At the same time, the SIMM model developed by International Swaps and Derivatives Association(ISDA) is a common methodology for calculating the initial margin for uncleared OTC derivatives. It also relies on asset price simulations to calculate potential portfolio risks and imposes several haircuts on the market values of riskier securities. 

In Blink, we use SPAN/SIMM scenario analysis of the underlying and haircuts approach as a starting point for Margin Engine design. As Blink Margin Engine relies upon heavy computations and scenario analysis, it is impossible to put these calculations on a blockchain, which is why it is a centralized system. In the meantime, it is still designed to be used as a standard in the DeFi space for decision-making regarding portfolio risk and maintenance margin calculation. 

\section {Architecture}

Blink Margin Engine analyses the portfolio of whitelisted (supported) assets, tokens, and DEXs in real time. 
The methodology is designed to be configured for any universe/Liquidation Routing/Asset Price Information. However, we will use a whitelisted universe from Blink Protocol. The list of supported assets, pools, and DEXs for Blink Protocol can be found in Appendix 1.

The system is designed to give its user an answer to the following questions:
  \begin{enumerate}
  \item What is the current portfolio Net Asset Value (NAV) relative to a borrowed asset?
  \item What is the stress-tested NAV? What is risk-factor and maintenance margin value right now?
  \item What is the market impact/cost of liquidating token X for token Y on any supported DEX considering market impact?
  \item What is the gas/bridging cost of liquidating a portfolio using defined Liquidation Routing?
  \item What is the total liquidity pool position value relative to a borrowed asset considering fees to be claimed?
  \item What actions and trades need to be completed to liquidate a portfolio considering the Liquidation Routing scheme if a portfolio should be liquidated right now?
  \end{enumerate}


\subsection{Reference Price and Liquidation Routing}
The key purpose of any margin engine is to calculate the initial and maintenance margin at any given time. If the trader's portfolio can not satisfy maintenance margin requirements -  they get a margin call, and the portfolio is liquidated. As the liquidation process is quite specific in the DeFi space, especially in a multi-chain environment, market impact, gas, and bridging costs should also be included in margin calculations to detect liquidation levels.

First, we need to define \textbf {Reference Price} - the price of an asset X relative to the \textbf{Leverage Asset}. Blink Margin Engine uses ChainLink Oracles for that. For example, if a trader took USDC leverage on Ethereum to open a leveraged position in ETH, then ETH/USDC (Ethereum) Chainlink Oracle is used to determine Reference Price at any given time. Secondly, we introduce \textbf {Liquidation Routing} - the weight of each DEX in the liquidation process if a margin call happens. The Liquidation Routing table for Blink Protocol can be found in Appendix 2. Taking the example above, in Blink Protocol, for an open ETH position, if USDC is used as leverage, 70\% of ETH will be converted back into USDC on Uniswap, and 30\% will be swapped using 1INCH. We need to swap each token from the user's portfolio during liquidation into a leverage asset to bring it back into the liquidity pool where it was taken. Both Reference Price and Liquidation Routing are used in \textbf{Mark Price} calculations.

\subsection{Mark Price}

Mark Price is used to get an estimate of currently open position values. As Mark Price is used to estimate current position value considering unrealized profit and loss, it should consider current market liquidity for a specific token pair (asset/borrowed asset). That is why we introduce \textbf{Market Impact Factor}. Market Impact Factor estimates the market impact for any DEX from Liquidation Routing if a trader's position in token X is liquidated. That is why Market Impact Factor is a function of:
  \begin{enumerate}
  \item DEX and current DEX liquidity.
  \item Trader's position size in a token. Big, concentrated positions will have a bigger market impact. 	
  \end{enumerate}

At any given time $t$ for a token $i$, Mark Price is \begin{center} $\text{Mark Price}^{i}_{t}=\text{Reference Price}^{i}_{t}\times[1-\sum_{j=1}^{N}  w_{j}\times f_{t}^{j}\times(1+MarketImpactPct)]$ \end{center}
where

\begin{itemize}
\item $\text{Reference Price}^{i}_{t}$ - ChainLink Oracle price of a trading pair token/borrowed asset.
\item $w_{j}$ - the percentage of DEX $j$ in Liquidation Routing of a token $i$ (for example, in Blink Protocol, for ETH if the Borrowed asset is DAI,  Uniswap weights 70\%).
\item $N$ - number of exchanges in Liquidation Routing for a token $i$ 
\item $f_{t}^{j}$ - Market Impact Factor for DEX $j$ on token $i$.
	
\end{itemize}


To give even more margin of safety, Blink Margin Engine inflates Market Impact Factor calculations by $MarketImpactPct$ = 15\% to consider unexpected market impact during liquidations. 

As we have previously said, $f_{t}^{j} = F(\widetilde{E}, S, L_{t})$ where $\widetilde{E}$ defines DEX and how execution is handled on a particular DEX (AMM, Order-Book, RFQ), $S$ is size of a position, $L_{t}$ is liquidity on a given DEX at time $t$. Introducing the Market Impact Factor (MIF) gives us the flexibility in estimating the cost of liquidation where Liquidation Routing may consist of both AMM and Order-Book-based exchanges.


\subsection{Estimating Market Impact Factor for Order-Book-based Exchanges}

There are various well-known models of quantifying market impact in order-book markets such as TODO refer! Almgren (2005), Kissel (2004), Kyle. However, none of them are used in Blink Margin Engine. The reason for that is a fact that in market impact estimates, the Engine needs to answer, "By how much liquidating a large order size with a single market order will shift the price from the equilibrium?". This decision is mostly driven by the fact that the liquidation process on Blink Protocol is done through one order for each DEX in Liquidation Route. Once the liquidation process in the protocol becomes more advanced (splitting orders into smaller parts, VWAP, and other advanced execution algorithms), - market impact estimation will be changed accordingly.  

Blink Margin Engine uses a "walking over" limit order book approach to estimate Market Impact Factor. We use 5 levels of the limit order book to estimate the average execution price: $$ \text{Average Execution Price}_{t} = \frac{\sum\limits_{i=1}^{5} (P_i \times Q_i)}{\sum\limits_{i=1}^{5} Q_i} $$

In this formula, $P_i$ represents the price at which the market order was executed at level $i$ of the limit order book, and $Q_i$ represents the number of shares executed at that level. The summation is taken over all $n$ levels of the limit order book that were walked over by the market order. The numerator calculates the total value of shares executed at each price level, and the denominator calculates the total quantity of shares executed. The resulting value gives the average price per share at which the market order was executed.

$Q_i$ depends on liquidity at a particular level and liquidity on previous levels of the limit order book. Suppose we have the following limit order book for ETH/DAI pair shown in Table 1.

\begin{table}[h]
\centering
\begin{tabular}{| c | c |}
\hline
 Buy orders & Sell orders \\
\hline
 100  at 1000.00 & 200  at 1005 \\
 50  at 995 & 150  at 1010 \\
 75  at 990 & 100  at 1020 \\
 25  at 985 & 50  at 1030 \\
 50  at 980 & 25  at 1035 \\
\hline
\end{tabular}
\end{table}

Now let's consider a market order to sell 150 ETH. The market order "walks over" the limit order book, executing at each level until filled.
The first 100 ETH of the market order will fill the first level of the buy side of the order book at a price of $1000.00$, which exhausts this level of the order book.
The remaining 100 ETH of the market order will then fill the next level of the buy side of the order book at a price of $995$, which exhausts this level of the order book.
The market order is now filled, having sold 150 ETH at an average executing price of:
$$\frac{(100 \times 1000) + (50 \times 995)}{150} = 998.3$$



So the average executing price of the market order that walked over levels of the limit order book was $998.3$.

Finally, the Market Impact Factor is the absolute percent difference between Average Execution Price and Mid-Price, $V_{t} = (bid_{t} + ask_{t})/2$: $$f_{t} = |\frac{V_{t}}{\text{Average Execution Price}_{t}}}-1|$$
In our example, the Market Impact Factor equals $|\frac{(1000+1005)/2}{998.3}-1| = 0.0042$



\subsection{Estimating Market Impact Factor for AMM Exchanges}

\subsection{Estimating Market Impact Factor for DEX Aggregators(1INCH)}
Aggregators are the DeFi equivalent of smart order routing systems. Whenever a token trades in multiple pools, aggregators will buy the token across all pools to minimize the price impact on each one of them. Instead of spreading the trade over time in a single market, this order executes at once, spread over many possible markets. Aggregators also command substantially higher gas costs than a single trade, similar to splitting trades manually. TODO reference https://research.paradigm.xyz/amm-price-impact
TODO: add how be determine price impact on 1INCH.


\subsection{Mark Price Calculation Example}
Imagine a trader borrowed 2000 USDT to buy 1 ETH. At the time $t_{1}$, the Reference Price of ETH/USDT is 1000 USDT. Using the Liquidation Routing from Blink Protocol, we can see that if liquidated:
\begin{itemize}
	\item 70$\%$ of ETH will be liquidated through Uniswap.
	\item 30$\%$ of ETH will be liquidated through 1INCH.
\end{itemize}

Using the algorithms described in 3.4 and 3.5, we may define that the market impact factor for for liquidating:
\begin{itemize}
	\item 0.7 ETH on Uniswap is 0.03$\%$.
	\item 3.7 ETH on 1INCH is 0.01$\%$.
\end{itemize}

Which results in a Mark Price of ETH relative to USDT:
$$1000*(1-0.7*0.0003*1.15-0.3*0.0001*1.15)=999.724$$

\section{Portfolio Value Calculations}

The first step in calculating portfolio maintenance margin and a liquidation risk factor is to estimate each portfolio position's value separately.

\subsection{Token Price Calculations}

For whitelisted tokens, we use Mark Price to estimate the price of a token relative to a borrowed asset, however, several tokens have a \textbf{valuation haircut}:

\begin{itemize}
	\item stETH, wstETH: -5$\%$
	\item CRV: -15$\%$
\end{itemize}


For example, if the Mark Price of CRV/USDT is 1.088 and a user has 100 CRV in their portfolio in Blink Finance, the value of the position will be:
$$1.088 * 100 * (1-0.15)= 92.48$$


\subsection{Liquidity Pools Calculations}
Liquidity pools consist of staked assets and fees (rewards) to be claimed. If a user claims fees - their price is calculated by the Token Price Calculations method only if a fee token is whitelisted inside Blink Margin Engine. Non-whitelisted tokens are not included in portfolio value calculations.
Staked assets are calculated using the Token Price Calculations methodology. 

Unclaimed rewards (if rewards tokens are whitelisted by Blink) are calculated using the Token Price Calculations algorithm with a \textbf{haircut of 50$\%$}.

For example, if a user has Uniswap V3 position with \textbf{5 ETH and 10 000 USDT} and \textbf{0.7 ETH and 100 USDT in unclaimed rewards} to be claimed with the borrowed asset is USDT, and ETH/USDT Mark Price is 1000 then its position value is:
$$(5 * 1000 + 10000) + (0.7*1000+100)*0.5= 15400$$

\subsection{Lending Protocols Calculations}
For positions where a trader deposits collateral into a lending protocol (AAVE, Compound), a position value is calculated using Token Price Calculations by mapping the number of protocol tokens (aWETH, aUSDT, cUSDC, etc.) to collateral value. 


\subsection{Aggregating Portfolio by Assets}
As previously discussed, position value significantly influences the Market Impact Factor in Mark Price calculations, so we must aggregate portfolio positions by tokens. 

So, the position value in a token is the sum of all tokens on all chains, tokens in liquidity pools, and fees to be claimed in this token. For example, a trader has a portfolio built by borrowing USDC:
\begin{itemize}
	\item 10 wETH, 5 wBTC on Ethereum.
	\item 7 wETH, 10 000 USDT on BSC.
	\item Uniswap V3 position (10 wETH x 6000 DAI) + fees (3 wETH, 200 DAI) on Ethereum.
	\item Curve LP position in (5000 USDT/5000 USDC) and fees to be claimed (100 CRV, 200 USDT, 5000 USDC).
\end{itemize}

As a result of aggregation we have:
\begin{itemize}
	\item wETH: 10+7+10+3 = 30 wETH.
	\item wBTC: 5 wBTC.
	\item USDT: 10000 + 5000 + 200 = 15200 USDT.
	\item DAI: 6000 + 200 = 6200 DAI.
	\item USDC: 5000 + 5000 = 10000 USDC.
	\item CRV: 100 CRV.
\end{itemize}

The aggregated token values are now used as inputs to calculate the Market Impact Factors. As we can see after aggregation we have a relatively big position in wETH, which is why its the Market Impact Factor will influence the Mark Price heavily due to its big value.

For borrowed asset in a portfolio (USDC in the example above), MIF equals 0 as no swap activities are needed during liquidation. Only a return to LP is needed.

\section{Adjusting Portfolio Value for Gas/Bridging Fees}
This part of portfolio risk factor calculation applies only if the protocol/leverage entity intends to liquidate bad debt positions. In Blink Protocol, for example, the liquidation is done through Liquidation Routing without relying on external liquidators. The adjustment is needed as gas and bridging costs may significantly impact the overall liquidation cost. There are 4 types of actions that incur gas costs:

\begin{itemize}
	\item Asset swap (when we need to swap assets back to borrowed asset).
	\item Extract liquidity from a pool/close debt on a lending protocol.
	\item Bridging.
	\item Return funds to LP.
\end{itemize}

For tokens bridging fees arise if a token is not on the chain defined by the Liquidation Routing. For pools, gas fees for extracting liquidity/claiming fees are added.

From the example above:
\begin{itemize}
	\item Bridge+gas fees: 7wETH, 10 000 USDT from BSC to Ethereum.
	\item Gas fees: extracting liquidity from Uniswap V3, Curve.
	\item Gas fees: swapping 30 wETH, 5wBTC, 15200 USDT, 6200 DAI,  100 CRV.
	\item Gas fees: return USDC to Liquidity Pool.
\end{itemize}

Finally, the process of Portfolio Value Calculation can be described as:
\begin{enumerate}
	\item Aggregate portfolio positions by tokens.
	\item Define the Market Impact Factor as a function of position value for each token in a portfolio.
	\item Get tokens Mark Prices.
	\item Get the sum of token prices/liquidity pools/fees prices using Mark Prices and apply haircuts where needed.
	\item Get the sum of assets and adjust it for gas/bridging fees.
\end{enumerate}

\section{Portfolio Stress-Test Scenario Analysis}
Before calculating the \textbf{Risk Factor} used to determine if a user should be liquidated, we need to calculate \textbf{Stress Tested Portfolio Value}. Let’s use the example from above to show how calculations work in this case. A user has a portfolio with borrowed asset USDC:

\begin{itemize}
	\item 10 wETH, 5 wBTC on Ethereum.
	\item 7 wETH, 10 000 USDT on BSC.
	\item Uniswap V3 position (10 ETH x 6000 DAI) + fees (3 ETH, 200 DAI) on Ethereum.
	\item Curve LP position in (5000 USDT/5000 USDC) and fees to be claimed (100 CRV, 200 USDT, 5000 USDC).
\end{itemize}

Mark Prices are:

\begin{itemize}
	\item wETH/USDC: 1000
	\item wBTC/USDC: 10000
	\item DAI/USDC: 0.99
	\item USDT/USDC: 0.99
	\item CRV/USDC: 1.08
\end{itemize}

For all assets/LP positions held in the user's portfolio, we want to understand \textbf{“What happens to a position if its Mark Price drops by X$\%$ relative to the Leverage Asset?”}. 

\subsection{Prepare Scenario Analysis Matrix}
We stress-test available assets based on either exchange rate risk or de-pegging (for stablecoins, ETH/stETH) risk. Based on a token risk and borrowed asset, we apply a different rate of change in the scenario matrix. \textbf{Scenario Matrix}, $\boldsymbol{\Omega}$, is the matrix where each element represents a $\%$ change drop of the price of asset $i$ relative to borrowed asset $j$. Scenario Matrix is a configurable parameter of Blink Margin Engine; in our example, we will use the matrix from Blink Protocol:

$$\resizebox{\textwidth}{!}{\begin{tabular}{|l|c|c|c|c|c|c|c|c|c|c|c|c|c|}
  \hline
   & wETH(ETH) & wstETH & wBTC & DAI & USDT & USDC & FRAX & CRV & FXS & LIDO & MATIC \\
  \hline
  wETH(ETH) & - & -10$\%$ & -10$\%$ & -30$\%$ & -30$\%$ & -30$\%$ & -30$\%$ & -30$\%$ & -30$\%$ & -30$\%$ & -30$\%$  \\
  wstETH &  & - & -30$\%$ & -30$\%$ & -30$\%$ & -30$\%$ & -30$\%$ & -30$\%$ & -30$\%$ & -30$\%$ & -30$\%$\\
  wBTC &  &  & - & -30$\%$ & -30$\%$ & -30$\%$ & -30$\%$ & -30$\%$ & -30$\%$ & -30$\%$ & -30$\%$ \\
  DAI &  &  &  & - & -10$\%$ & -10$\%$ & -10$\%$ & -30$\%$ & -30$\%$ & -30$\%$ & -30$\%$ \\
  USDT &  &  &  &  & - & -10$\%$ & -10$\%$ & -30$\%$ & -30$\%$ & -30$\%$ & -30$\%$\\
 USDC &  &  &  &  & -10$\%$ & - & -10$\%$ & -30$\%$ & -30$\%$ & -30$\%$ & -30$\%$\\
  \hline
\end{tabular}}$$

\subsection{Get stress-tested values for portfolio tokens}
Based on borrowed assets and Mark Price for each token from the user’s portfolio, we get the token value using Mark Price adjusted for the value from the Scenario Matrix. The portfolio token’s values in USDC are:
\begin{itemize}
	\item 10 wETH [10 * 1000 * (1-0.3) = 7000], 5wBTC [5*10000* (1-0.3) = 35000] on Ethereum.
	\item 7 wETH [7 * 1000 * (1-0.3) = 4900], 10 000 USDT [10000 * 0.99 * (1-0.1) = 8910] on BSC.
\end{itemize}


\subsection{Get stress-tested values for LP positions}
For LP positions we need to consider not only delta risk (change of an asset price) but also gamma risk (Impermanent Loss). Taking the Impermanent Loss into consideration is extremely important taking into account non-linear nature of Liquidity Pools, especially those exposed to concentrated liquidity (Uniswap V3). When stress-testing LP positions the algorithm in the following: 
\begin{enumerate}
	\item Calculate the updated token price ratio considering the updated exchange rate relative to a borrowed asset. 
	\item Get updated LP position quantities under the new token X/token Y price ratio.
	\item Get updated price of LP position denominated in a borrowed asset.
\end{enumerate}

\subsubsection{Estimating position quantities for Uniswap V2}
Using the constant product formula  $x*y=k$ where $x$ is the number of tokens X in the pool and $y$ is the number of token Y, we can get the estimate of expected token quantities inside of LP as a function of $k$ and the exchange rate $p_{t}$ between $x$ and $y$:
$$x_{t} = \sqrt{k/p_{t}}$$ $$y_{t}=\sqrt{k*p_{t}}$$ 
Secondly, for each LP position, we keep the percentage stake, $s_{t}$, in a pool for a given position which is reflected in "liquidity tokens" provided to a user. If others subsequently add/withdraw coins, new liquidity tokens are minted/burned such that everyone’s relative percentage share of the liquidity pool remains the same. TODO reference https://docs.uniswap.org/contracts/v2/concepts/advanced-topics/understanding-returns

Finally, a stress-tested number of tokens for a user $(x^{*}, y^{*})$ is: $$x^{*} = s_t*\sqrt{k/p^{*}$$ $$y^{*} = s_t{}*\sqrt{k*p^{*}$$
Where $p^{*}$ is the price ratio of stress-tested token X to token Y prices: $$p^{*} = \frac{\Omega(\text{borrowed asset, token X)}}{\Omega\text{(borrowed asset, token Y)}}$$


\subsubsection{Estimating values for Uniswap V3}
For concentrated liquidity, price ranges may significantly impact updated token quantities under a stress-test scenario. As a result, Uniswap V3 LP positions are exposed to a bigger gamma risk compared to V2. TODO: add math.



\subsection{Get stress tested fees values}
 For token values, we use the same approach as we did with tokens + haircut adjustment. 
 \begin{itemize}
 	\item Uniswap V3 fees: 3 ETH, 200 DAI = 3 * 700 * 0.5 + 200 * 0.99 * 0.5 = 1149 USDC.
 	\item Curve pool fees: 100 CRV, 200 USDT, 5000 USDC = 100 * 0.756 * 0.5 + 200 * 0.99 * 0.5 + 5000 * 0.5 = 2636.8
 \end{itemize}
 
 \subsection{Find the total positions sum and adjust for gas/bridging fees}
 
 Once we get adjusted for stress test position values in borrowed asset, we find the total sum and adjust the value for gas/bridging fees.
 
 \section{Risk Factor Calculation}
 Finally, we can calculate the \textbf{Risk Factor} used to determine if a trader’s position should be liquidated. 
 
 $$RiskFactor_{t}=\frac{StressPortfolioValue_{t}-LiquidationPremium -LiquidationFee-Interest(t)}{BorrowedAssetAmount}$$
 
 \textbf{Liquidation Premium} is an extra margin of safety Blink Protocol takes to liquidate a user for unpredicted events which may occur, while \textbf{Liquidation Fee} is a fee that is deposited to populate Blink Insurance Fund. In Blink Protocol, $LiquidationPremimum$ and $LiquidationFee$ are set to $10\%$. $Interest(t)$ refers to cumulative accrued interest for using leverage. A user is liquidated if their Risk Factor \textbf{drops below 1}. In our example, assuming th


\section {References}
\begin{enumerate}
	\item \textit{Commitments of Traders.}.Commodity Futures Trading Commission. Accessed January 27, 2021. 
	
	https://www.cftc.gov/MarketReports/CommitmentsofTraders/index.htm. 
	\item Lopez De Prado,M.2018.\textit{Advances in Financial Machine Learning}. Hoboken: John Wiley \&\ Sons. 
	\item Lopez De Prado, M.2020b.\textit{Machine Learning for Asset Managers}. Cambridge: Cambridge University Press.
	\item Kahn, A. D.1986.\textit{“Conformity with Large Speculators: A Test of Efficiency in the Grain Futures Market.”} Atlantic Econ. J. 14: 51-55.
	\item Wang, C.2001.\textit{“Investor Sentiment and Return Predictability in Agricultural Futures Markets.”} J. Futures Mkts. 21,10: 929–952.
	\item Bryant, H., D. A. Bessler, and M. S. Haigh.2006.\textit{“Causality in Futures Markets”}. J. Futures Mkts. 26,11: 1039–1057.
	\item Sanders, D. R., Irwin, S. H. and Merrin, R. P. 2009. \textit{"Smart Money: The Forecasting Ability of CFTC Large Traders in Agricultural Futures Markets"}. Journal of Agricultural and Resource Economics 34(2):276–296.
	\item S., Keenan Mark J. \textit{Advanced Positioning, Flow and Sentiment Analysis in Commodity Markets Bridging Fundamental and Technical Analysis.} Chichester, West Sussex, United Kingdom: Wiley, 2020. 
	\item Lundberg S. M., and S. I. Lee. \textit{A Unified Approach to Interpreting Model Predictions.} NeurIPS Proceedings: Advances in Neural Information Processing Systems, 2017.
	
\end{enumerate}

\end{document}